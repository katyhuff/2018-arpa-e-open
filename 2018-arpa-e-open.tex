%        File: 2018-arpa-e-open.tex
%     Created: Tue Feb 06 01:00 PM 2018 C
% Last Change: Tue Feb 06 01:00 PM 2018 C
%
\documentclass[letterpaper,12pt]{article}
\usepackage[top=1in,bottom=1.0in,left=1.0in,right=1.0in,headsep=0.5in]{geometry}
\usepackage{verbatim}
\usepackage{amssymb}
\usepackage{graphicx}
\usepackage{longtable}
\usepackage{amsfonts}
\usepackage{amsmath}
\usepackage[hidelinks]{hyperref}
\def\thesection       {\arabic{section}}
\def\thesubsection     {\thesection.\alph{subsection}}

\author{Kathryn Huff, Madicken Munk\\
        \href{mailto:kdhuff@illinois.edu}{\texttt{kdhuff@illinois.edu}},
        \href{mailto:mmunk2@illinois.edu}{\texttt{mmunk2@illinois.edu}}
}

\date{}
\title{Machine Learning Acceleration of Monte Carlo Neutron Transport}


%
% The ARPA-E assigned Control Number, the Lead Organization Name, and the
% Principal Investigator’s Last Name must be prominently displayed on the upper
% right corner of the header of every page. Page numbers must be included in the
%footer of every page.
%
\usepackage{fancyhdr}
\pagestyle{fancy}
\fancyhf{}
\rhead{\#XXX-XXX\\
       U. Illinois\\
       Huff}
\cfoot{\thepage}
\begin{document}
\maketitle

\thispagestyle{fancy}

\begin{description}
    \item[Technical Category] 4D, Nuclear Power Generation And Materials
    \item[Duration] 36 Months
    \item[Amount] \$500,000
\end{description}
\section{Concept Summary}


Thiw work will implement a proof-of-concept python package demonstrating
acceleration of monte carlo methods (in particular, monte carlo for neutral
particle transport) using machine learning techniques. 
Next, this work will implement a few potential acceleration methods inspired by machine
learning and optimization techniques in the literature. 
Finally, these acceleration methods will be tested and compared to one another for simple
problems. This work will emphasize implementation of best practices in
scientific computing, including integrating documentation, implementing unit
tests, investigating scalability, and developing demonstration jupyter
notebooks. 

\section{Innovation and Impact}
\section{Proposed Work}
\section{Team Organization and Capabilities}
\nocite{*}
\pagebreak
\bibliographystyle{ieeetr}
\bibliography{2018-arpa-e-open}
\end{document}


