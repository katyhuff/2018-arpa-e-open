%        File: 2018-arpa-e-open.tex
%     Created: Tue Feb 06 01:00 PM 2018 C
% Last Change: Tue Feb 06 01:00 PM 2018 C
%
\documentclass[letterpaper,12pt]{article}
\usepackage[top=1in,bottom=1.0in,left=1.0in,right=1.0in,headsep=0.5in]{geometry}
\usepackage{verbatim}
\usepackage{amssymb}
\usepackage{graphicx}
\usepackage{longtable}
\usepackage{amsfonts}
\usepackage{amsmath}
\usepackage[hidelinks]{hyperref}
\def\thesection       {\arabic{section}}
\def\thesubsection     {\thesection.\alph{subsection}}

\author{Kathryn Huff, Madicken Munk\\
        \href{mailto:kdhuff@illinois.edu}{\texttt{kdhuff@illinois.edu}},
        \href{mailto:mmunk2@illinois.edu}{\texttt{mmunk2@illinois.edu}}
}

\date{}
\title{Machine Learning Acceleration of Monte Carlo Neutron Transport}


%
% The ARPA-E assigned Control Number, the Lead Organization Name, and the
% Principal Investigator’s Last Name must be prominently displayed on the upper
% right corner of the header of every page. Page numbers must be included in the
%footer of every page.
%
\usepackage{fancyhdr}
\pagestyle{fancy}
\fancyhf{}
\rhead{\#XXX-XXX\\
       U. Illinois\\
       Huff}
\cfoot{\thepage}
\begin{document}
\maketitle

\thispagestyle{fancy}

\begin{description}
    \item[Technical Category] 4D, Nuclear Power Generation And Materials
    \item[Duration] 36 Months
    \item[Amount] \$500,000
\end{description}

\section{Concept Summary}
%    Describe the proposed concept with minimal jargon, and explain how it
%               addresses the Program Objectives of the FOA.


This work will implement a proof-of-concept python package demonstrating
acceleration of monte carlo methods (in particular, monte carlo for neutral
particle transport) using machine learning techniques. 
Next, this work will implement a few potential acceleration methods inspired by machine
learning and optimization techniques in the literature. 
Finally, these acceleration methods will be tested and compared to one another for simple
problems relevant to advanced nuclear reactor safety and design. 
This work will emphasize implementation of best practices in
scientific computing, including integrating documentation, implementing unit
tests, investigating scalability, and developing demonstration jupyter
notebooks. 

\section{Innovation and Impact}
%    Clearly identify the problem to be solved with the proposed technology 
%              concept.
%    Describe how the proposed effort represents an innovative and potentially 
%              transformational solution to the technical challenges posed by the FOA.
%    Explain the concept's potential to be disruptive compared to existing or 
%              emerging technologies. 
%    Describe how the concept will have a positive impact on at least one of 
%              the ARPA-E mission areas in Section I.A of the FOA.
%    To the extent possible, provide quantitative metrics in a table that 
%              compares the proposed technology concept to current and emerging 
%              technologies and to the appropriate Technology Category in Section I.D of 
%              the FOA.




\section{Proposed Work}
%    Describe the final deliverable(s) for the project and the overall
%             technical approach used to achieve project objectives.
%    Discuss alternative approaches considered, if any, and why the proposed
%             approach is most appropriate for the project objectives.
%    Describe the background, theory, simulation, modeling, experimental data,
%             or other sound engineering and scientific practices or principles that support
%             the proposed approach.  Provide specific examples of supporting data and/or
%             appropriate citations to the scientific and technical literature.
%    Describe why the proposed effort is a significant technical challenge and
%             the key technical risks to the project.  Does the approach require one or more
%             entirely new technical developments to succeed?  How will technical risk be
%             mitigated? 
%    Identify techno-economic challenges to be overcome for the proposed
%             technology to be commercially relevant. 
%    Estimated federal funds requested; total project cost including cost
%             sharing.



\section{Team Organization and Capabilities}
%    Indicate the roles and responsibilities of the organizations and key
%             personnel that comprise the Project Team.
%    Provide the name, position, and institution of each key team member and
%             describe in 1-2 sentences the skills and experience that he/she brings 
%             to the team.
%    Identify key capabilities provided by the organizations comprising the
%             Project Team and how those key capabilities will be used in the proposed effort

\nocite{*}
\pagebreak
\bibliographystyle{ieeetr}
\bibliography{2018-arpa-e-open}
\end{document}


