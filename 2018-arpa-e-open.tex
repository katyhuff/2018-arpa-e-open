%        File: 2018-arpa-e-open.tex
%     Created: Tue Feb 06 01:00 PM 2018 C
% Last Change: Tue Feb 06 01:00 PM 2018 C
%
\documentclass[letterpaper,12pt]{article}
\usepackage[top=1in,bottom=1.0in,left=1.0in,right=1.0in,headsep=0.5in]{geometry}
\usepackage{verbatim}
\usepackage{amssymb}
\usepackage{graphicx}
\usepackage{longtable}
\usepackage{amsfonts}
\usepackage{amsmath}
\usepackage{paralist}


%-----------------------------------------------------------

\usepackage{graphicx}  % Required for including images

\usepackage{tabularx}
\newcolumntype{b}{X}
\newcolumntype{s}{>{\hsize=.5\hsize}X}
\newcolumntype{m}{>{\hsize=.75\hsize}X}
\newcolumntype{z}{>{\hsize=.65\hsize}X}

\usepackage{booktabs} % Top and bottom rules for tables
\usepackage{xspace}
\usepackage{tikz}
\usepackage{chronology}
\usetikzlibrary{arrows.meta,backgrounds}

\usetikzlibrary{positioning, arrows, decorations, shapes, calc}
% Define block styles
\tikzstyle{decision} = [diamond, draw, fill=blue!20,
text width=4.5em, text badly centered, node distance=3cm, inner sep=0pt]

\tikzstyle{const} = [rectangle, draw, text centered, fill=orange!20]
\tikzstyle{data} = [rectangle, draw, text centered, fill=green!20]

\tikzstyle{block} = [rectangle, draw, text centered, fill=blue!20]
\tikzstyle{line} = [draw, -latex']
\tikzstyle{cloud} = [draw, ellipse,fill=red!20, node distance=6em,
minimum height=2em]


\usetikzlibrary{shapes.multipart}
\usetikzlibrary{positioning}


%-----------------------------------------------------------


\usepackage[hidelinks]{hyperref}
\def\thesection       {\arabic{section}}
\def\thesubsection     {\thesection.\alph{subsection}}

\author{Kathryn Huff, Madicken Munk\\
        \href{mailto:kdhuff@illinois.edu}{\texttt{kdhuff@illinois.edu}},
        \href{mailto:mmunk2@illinois.edu}{\texttt{mmunk2@illinois.edu}}
}

\date{}
\title{Machine Learning Acceleration of Monte Carlo Neutron Transport}


%
% The ARPA-E assigned Control Number, the Lead Organization Name, and the
% Principal Investigator’s Last Name must be prominently displayed on the upper
% right corner of the header of every page. Page numbers must be included in the
%footer of every page.
%
\usepackage{fancyhdr}
\pagestyle{fancy}
\fancyhf{}
\rhead{\#XXX-XXX\\
       U. Illinois\\
       Huff}
\cfoot{\thepage}
\begin{document}
\maketitle

\thispagestyle{fancy}
\begin{center}
\begin{description}
    \item[Technical Category] 4D, Nuclear Power Generation And Materials
    \item[Duration] 36 Months
    \item[Amount] \$500,000
\end{description}
\end{center}

\section{Concept Summary}
%    Describe the proposed concept with minimal jargon, and explain how it
%               addresses the Program Objectives of the FOA.


This work will implement a proof-of-concept python package demonstrating
acceleration of Monte Carlo methods (in particular, Monte Carlo for neutral
particle transport) using machine learning techniques.
Next, this work will implement a few potential acceleration methods inspired by machine
learning and optimization techniques in the literature.
Finally, these acceleration methods will be tested and compared to one another for simple
problems relevant to advanced nuclear reactor safety and design.
This work will emphasize implementation of best practices in
scientific computing, including integrating documentation, implementing unit
tests, investigating scalability, and developing demonstration jupyter
notebooks.

\section{Innovation and Impact}
%    Clearly identify the problem to be solved with the proposed technology
%              concept.
%    Describe how the proposed effort represents an innovative and potentially
%              transformational solution to the technical challenges posed by the FOA.
Hybrid methods to reduce variance in neutral particle transport in nuclear
applications tend to be application-specific. In particular, methods generally
have been developed for fixed-source or eigenvalue calculations, but not both.
This method can be used on solution types, offering the potential for
a large application space and a broad applicability. Further, many hybrid
methods use a deterministic solution of the inverse solution to estimate problem
importance\cite{wagner_automated_1998, wagner_automated_2009,
haghighat_monte_2003, zhang_adjoint-based_2011, hoogenboom_optimum_1979} . Though methods exist that automate importance map
generation\cite{hendricks_mcnp_1985, burn_optimizing_2014,
  van_wijk_easy_2011, wagner_automated_1998, wagner_automated_2009,
haghighat_monte_2003, zhang_adjoint-based_2011, hoogenboom_optimum_1979},
they still require user knowledge to choose deterministic
parameter choices (space- and energy- mesh size, quadrature type and order,
etc.). Selecting these is a non-trivial task and dramatically affects performance
across hybrid methods. This work will automate the importance map parameter selection
and, additionally, require exclusively Monte Carlo to generate it. This is more
feasible from the user perspective and improves energy efficiency by reducing
the Monte Carlo runtime.
%    Explain the concept's potential to be disruptive compared to existing or
%              emerging technologies.
%    Describe how the concept will have a positive impact on at least one of
%              the ARPA-E mission areas in Section I.A of the FOA.
%    To the extent possible, provide quantitative metrics in a table that
%              compares the proposed technology concept to current and emerging
%              technologies and to the appropriate Technology Category in Section I.D of
%              the FOA.

\begin{table}
  \centering
    \begin{tabular}{ | l | c | }
     \hline
     Metric & Execution Time \\ \hline
     Monte Carlo Runtime & 50\%  \\
     Hybrid Method Runtime & 20\%-90\% \\
     User Time & 10\% \\
     \hline
    \end{tabular}
    \caption{Table summarizing realms of improvement and the fractional
      execution times that may be observed with Machine Learning Monte Carlo.
      Here the execution time is calculated by taking
    the ratio of the proposed Machine Learning Monte Carlo method and the
    industry standard method. Monte Carlo runtime compares the Machine
  Learning Monte Carlo to standard unbiased Monte Carlo. Hybrid runtime
  compares the time of execution of the Machine Learning Monte Carlo to a
  standard deterministic-Monte Carlo run. User time compares a user's time spent
adjusting deterministic paramater choices and setting up the hybrid run input.}
\end{table}


\section{Proposed Work}
%    Describe the final deliverable(s) for the project and the overall
%             technical approach used to achieve project objectives.


%=======================================================
% FIGURE BEGINS
%=========================================================


\newlength{\figwidth}
\setlength{\figwidth}{2.75cm}

\begin{figure}
    \centering
    \scalebox{1.0}{
                \begin{tikzpicture}[>={Latex[width=3mm,length=3mm]},
                                node distance=\figwidth,
                                on grid,
                                align=center,
                                bend angle=60,
                                auto]
        % Place nodes
        \node [const] (openmc) {\textbf{OpenMC}};
        \node [data, above=\figwidth of openmc] (geom) {Geometry\\Definitions};
        \node [data, right=\figwidth of geom] (mat) {Material\\Definitions};
        \node [data, left=\figwidth of geom] (xs) {Cross\\Section Data};
        \node [data, right=\figwidth of mat] (tallyspecs) {Tally\\Definitions};
        \node [data, right=\figwidth of tallyspecs] (nparticles) {N\\(particles)};
        \node [data, right=\figwidth of nparticles] (srcdist) {Source\\Distribution\\or Biasing};
        \node [cloud, below=\figwidth of openmc] (tally) {Other\\Tallies};
        \node [cloud, left=\figwidth of tally] (flux) {Fluxes};
        \node [cloud, right=\figwidth of tally] (hist) {Particle\\Histories};
        \node [block, below=\figwidth of tally] (imp) {Importance};
        \node [block, right=2*\figwidth of openmc] (ww) {\texttt{Weight Windows}};
        \node [data, right=2*\figwidth of imp] (params) {ML Model\\Parameters};
        \node [block, above=\figwidth of params] (model) {\texttt{ML model}};
        \begin{scope}[on background layer]
        \draw[->, ultra thick] let \p1=($(openmc)-(ww)$) in (ww) -- +(0,\y1)-- +(openmc);
        \draw[->, ultra thick] (flux) -- (imp);
        \draw[->, ultra thick] (tally) -- (imp);
        \draw[->, ultra thick] (hist) -- (imp);
        \draw[->, ultra thick] (geom) -- (openmc);
        \draw[->, ultra thick] (xs) -- (openmc);
        \draw[->, ultra thick] (mat) -- (openmc);
        \draw[->, ultra thick] (tallyspecs) -- (openmc);
        \draw[->, ultra thick] (nparticles) -- (openmc);
        \draw[->, ultra thick] (srcdist) -- (openmc);
        \draw[->, ultra thick] (openmc) -- (tally);
        \draw[->, ultra thick] (openmc) -- (hist);
        \draw[->, ultra thick] (openmc) -- (flux);
        \draw[->, ultra thick] (imp) -- (model);
        \draw[->, ultra thick] (params) -- (model);
        \draw[dashed,->, ultra thick] (model) to [bend right] node [above right] {} (tallyspecs);
        \draw[dashed,->, ultra thick] (model) to [bend right] node [above right] {} (nparticles);
        \draw[dashed,->, ultra thick] (model) to [bend right] node [above right] {} (srcdist);
        \draw[->, ultra thick] (model) -- (ww);
        \path[->] (openmc) edge [ultra thick,out=330,in=0,looseness=10] 
        (openmc) node[below right=0.1\figwidth and 0.8\figwidth] 
        {\textbf{simulate}\\\textbf{N particles}};
        \end{scope}
        \end{tikzpicture}}
    \caption{Basic methodology for machine learning importance mapping.}
\end{figure}


%    Discuss alternative approaches considered, if any, and why the proposed
%             approach is most appropriate for the project objectives.
%    Describe the background, theory, simulation, modeling, experimental data,
%             or other sound engineering and scientific practices or principles that support
%             the proposed approach.  Provide specific examples of supporting data and/or
%             appropriate citations to the scientific and technical literature.
%    Describe why the proposed effort is a significant technical challenge and
%             the key technical risks to the project.  Does the approach require one or more
%             entirely new technical developments to succeed?  How will technical risk be
%             mitigated?
A number of attempts have been made to develop hybrid methods with broad
applicability across phase-space, such as the CADIS
methodology\cite{haghighat_monte_2003}. Many of the
existing methods that are successful use a deterministic transport
calculation on a coarse mesh solving the inverse problem to obtain an importance
estimate\cite{zhang_global_2014, zhang_adjoint-based_2011, haghighat_monte_2003,
hendricks_mcnp_1985}. Though this method will use machine learning in a novel space to
update Monte Carlo, the quality of the importance estimate will still be limited
by the quality of the results from the initial Monte Carlo simulation. The
problem physics (a very effective shield, a highly absorbing medium, etc.) will
influence particles' ability to penetrate through the problem phase-space, and
this may result in very noisy importance estimates in certain regions of the
problems\cite{van_wijk_easy_2011}. Further, this noise may exacerbate overfitting phenomena observed in
Machine Learning techniques. Because this method has a large phase-space
applicability, it may require a substantial computational burden to investigate
the method performance in different tally types (space, space-energy,
space-energy-angle).

Using Monte Carlo for full-core simulation of nuclear reactors is
computationally expensive\cite{martin_challenges_2012}. However, Monte Carlo is
attractive as a method because it is inherently continuous in space, energy, and
angle, potentially allowing for higher fidelity results in tally regions.
This method may reduce
the variance in the solution we seek but may not overcome the computational
limitations that exist for full-core reactor simulation with Monte Carlo. We plan to
characterize the method on a diverse set of application problems, ranging from
geometrically simple to complex and including either fixed-source or criticality
(eigenvalue) calculations.

A method that reduces the runtime for full-core reactor core calculations
presents the opportunity to offer the most energy savings in advanced reactor
simulation. Working designs for future reactors have numerous design iterations,
each requiring multiple Monte Carlo runs to simulate the evolution of the
reactor over a lifetime. Among the problems that will be
simulated with this method are:
\begin{compactitem}
  \item A molten salt reactor \cite{robertson_msre_1965}
  \item A pebble bed, fluorite salt-cooled, high temperature
    reactor \cite{andreades_design_2016}, and
  \item A molten salt, single-fluid, breeder
    reactor\cite{robertson_conceptual_1971}.
\end{compactitem}
In addition to energy savings, faster runtime will also allow for faster design
turnaround for different reactor point designs.
%    Identify techno-economic challenges to be overcome for the proposed
%             technology to be commercially relevant.
%    Estimated federal funds requested; total project cost including cost
%             sharing.
\paragraph{Budget} The project has a \$\textbf{XXXX} federal budget over a 24 month duration. Fuding at UIUC will support 3 graduate students, one postdoc, 8 weeks of faculty summer salary, and other expenses such as conference travel and publication page charges.


\section{Team Organization and Capabilities}
%    Indicate the roles and responsibilities of the organizations and key
%             personnel that comprise the Project Team.
%    Provide the name, position, and institution of each key team member and
%             describe in 1-2 sentences the skills and experience that he/she brings
%             to the team.

\paragraph{Kathryn Huff} is a Blue Waters Assistant Professor at the University
of Illinois at Urbana-Champaign (UIUC) department of Nuclear, Plasma, and
Radiological Engineering (NPRE). She leads the Advanced Reactors and Fuel
Cycles Research Group which focuses on development of nuclear engineering
software such as the Cyclus simulator, PyNE, and Moltres, a MOOSE application
for MSR physics. She will provide insight on scientific computing, the nuclear
fuel cycle, the Moltres tool, and object-oriented multiphysics modeling in
general.

\paragraph{Madicken Munk} is a Postdoctoral Research Scholar at the University
of Illinois at Urbana-Champaign (UIUC) in the National Center for Supercomputing
Applications (NCSA). She will provide expertise in computational methods
development, Monte Carlo radiation transport, and hybrid neutral-particle
transport. Madicken will serve as a reference and mentor on radiation transport
methods and will help oversee the method development.
%    Identify key capabilities provided by the organizations comprising the
%             Project Team and how those key capabilities will be used in the proposed effort

\paragraph{Key Capabilities}

UIUC is a leader in world class computing resources that will be available for compute-
intensive aspects of this work. In particular, the Blue Waters supercomputer is
one of the most powerful supercomputers in the world, and is the fastest
supercomputer on a university campus.
\pagebreak
\bibliographystyle{plain}
\bibliography{2018-arpa-e-open}
\end{document}


