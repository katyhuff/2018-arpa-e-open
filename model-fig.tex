
%=======================================================
% FIGURE BEGINS
%=========================================================


\newlength{\figwidth}
\setlength{\figwidth}{2.75cm}

\begin{figure}
    \centering
    \scalebox{0.6}{
                \begin{tikzpicture}[>={Latex[width=3mm,length=3mm]},
                                node distance=\figwidth,
                                on grid,
                                align=center,
                                bend angle=60,
                                auto]
        % Place nodes
        \node [const] (openmc) {\textbf{OpenMC}};
        \node [data, above=\figwidth of openmc] (geom) {Geometry\\Definitions};
        \node [data, right=\figwidth of geom] (mat) {Material\\Definitions};
        \node [data, left=\figwidth of geom] (xs) {Cross\\Section Data};
        \node [data, right=\figwidth of mat] (tallyspecs) {Tally\\Definitions};
        \node [data, right=\figwidth of tallyspecs] (nparticles) {N\\(particles)};
        \node [data, right=\figwidth of nparticles] (srcdist) {Source\\Distribution\\or Biasing};
        \node [cloud, below=\figwidth of openmc] (tally) {Other\\Tallies};
        \node [cloud, left=\figwidth of tally] (flux) {Fluxes};
        \node [cloud, right=\figwidth of tally] (hist) {Particle\\Histories};
        \node [block, below=\figwidth of tally] (imp) {Importance};
        \node [block, right=2*\figwidth of openmc] (ww) {\texttt{Weight Windows}};
        \node [data, right=2*\figwidth of imp] (params) {ML Model\\Parameters};
        \node [block, above=\figwidth of params] (model) {\texttt{ML model}};
        \begin{scope}[on background layer]
        \draw[->, ultra thick] let \p1=($(openmc)-(ww)$) in (ww) -- +(0,\y1)-- +(openmc);
        \draw[->, ultra thick] (flux) -- (imp);
        \draw[->, ultra thick] (tally) -- (imp);
        \draw[->, ultra thick] (hist) -- (imp);
        \draw[->, ultra thick] (geom) -- (openmc);
        \draw[->, ultra thick] (xs) -- (openmc);
        \draw[->, ultra thick] (mat) -- (openmc);
        \draw[->, ultra thick] (tallyspecs) -- (openmc);
        \draw[->, ultra thick] (nparticles) -- (openmc);
        \draw[->, ultra thick] (srcdist) -- (openmc);
        \draw[->, ultra thick] (openmc) -- (tally);
        \draw[->, ultra thick] (openmc) -- (hist);
        \draw[->, ultra thick] (openmc) -- (flux);
        \draw[->, ultra thick] (imp) -- (model);
        \draw[->, ultra thick] (params) -- (model);
        \draw[dashed,->, ultra thick] (model) to [bend right] node [above right] {} (tallyspecs);
        \draw[dashed,->, ultra thick] (model) to [bend right] node [above right] {} (nparticles);
        \draw[dashed,->, ultra thick] (model) to [bend right] node [above right] {} (srcdist);
        \draw[->, ultra thick] (model) -- (ww);
        \path[->] (openmc) edge [ultra thick,out=330,in=0,looseness=10] 
        (openmc) node[below right=0.1\figwidth and 0.8\figwidth] 
        {\textbf{simulate}\\\textbf{N particles}};
        \end{scope}
        \end{tikzpicture}}
    \caption{Methodology for machine learning accleration of monte carlo. Green 
            boxes indicate input data, red ovals indicate output data, and blue 
            boxes represent the core methods to be implemented in this work. 
            Dotted arrows represent areas of later extension and exploration.}
            \label{fig:model-fig}
\end{figure}

